\documentclass[12pt]{extarticle}
\usepackage[utf8]{inputenc}
\usepackage{cite}
\usepackage{amsmath}

\title{A Parallel Application of the Fourier Transformation}
\author{Justin Spidell -- Brett Sumser}
\date{October 2021}

\begin{document}
 
\maketitle
\newpage
\begin{abstract}

A Fourier transform is a mathematical transform decomposing functions based on space and time into functions based on spatial or temporal frequency. The Fourier transform is denoted by adding a circumflex to the symbol of a function:

\begin{align*}
f \rightarrow \hat{f}
\end{align*}

The Fourier transform is defined as:

\begin{align}
\hat{f} (\xi)= \int^{\infty}_{-\infty}f(x) e^{-2 \pi i x \xi}dx
\end{align}
Whereas the inverse Fourier transform is denoted as:

\begin{align}
{f} (x)= \int^{\infty}_{-\infty}\hat{f}(\xi) e^{2 \pi i x \xi}d\xi
\end{align}

\end{abstract}


\maketitle
\newpage
\section*{Introduction}

    The Fourier Transform is an important mathematical concept. It has applications
    in digital signal processing, convolution in neural networks, and even image recognition.

\maketitle
\newpage
\section*{Parallelization} 
    
    There are a few different directions to explore when developing a more
    parallized implementation of the Fourier Transform.
    
    This is a test citation. \cite{knuth:1984}


\bibliographystyle{plain}
\bibliography{refs} % Entries are in the &quot;refs.bib&quot; file</code></pre>

\end{document}